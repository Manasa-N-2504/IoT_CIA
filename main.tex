\documentclass{report}
\usepackage[utf8]{inputenc}
\usepackage{graphics}
\graphicspath{ {./images/} }

% width,height
\usepackage[a4paper, total={7in, 10in}]{geometry}

\title{INTERNET OF THINGS - RESEARCH PAPER SUMMARY}
\author{MANASA N}
\date{\today}

\begin{document}

    \begin{titlepage}
    \centering
        \vspace*{0.2cm}
        \Huge
        \textbf{INTERNET OF THINGS ARCHITECTURE AND PROTOCOLS}\\
        \vspace*{0.4cm}
        \Huge
        \textit{RESEARCH PAPER SUMMARY}\\
        \normalsize
        \vspace*{0.5cm}
        NAME : MANASA N\\
        \vspace{0.2cm}
        ROLL NO : 21011101073\\
        \vspace{0.2cm}
        AI and DS - B\\
        \vspace{1.2cm}
        \textbf{\underline {TITLE : SMART CITY DEVELOPMENT}}\\
        \vspace*{0.5cm}
        \includegraphics{Folder/pic1.png}\\
        \vspace*{1cm}
        DEPT : COMPUTER SCIENCE AND ENGINEERING\\
        \textbf{SHIV NADAR UNIVERSITY, CHENNAI}\\
        \includegraphics{Folder/logo3.jpg}  
\end{titlepage}
    
    \begin{center}
        \section*{SMART CITY DEVELOPMENT}
    \end{center}
    \large
    \section*{1. SUMMARY :}
    
    \textit{The paper begins by providing an overview of the concept of smart cities and the various technological components that make them possible - IoT. The authors then discuss the various \textbf{APPLICATIONS of IoT} in smart cities -\\ Health of buildings, Environmental Monitoring, Waste Management, Smart Parking, Smart Health, Navigation System, Smart Grid and Autonomous Driving.
    \begin{itemize}
        \item \textbf{Transportation:} Used to improve traffic flow, reduce congestion, and make transportation safer
        \item \textbf{Energy management:} Used to monitor and control energy consumption in buildings and homes, reducing waste and promoting energy efficiency
        \item \textbf{Public safety:} Used to improve public safety by detecting and responding to potential threats, such as crime and natural disasters
        \item \textbf{Environmental monitoring:} Used to measure and monitor air and water quality, and to track the movement of pollutants
        \item \textbf{Smart grid:} Enables two-way communication between connected devices and hardware that sense and respond to user demands
        \item \textbf{Autonomous Driving:} Enables cars to connect wirelessly to a cloud system
    \end{itemize}
    Next is the \textbf{LAYERS of IoT} that include - 
    \begin{enumerate}
        \item \textbf{Perception Layer}\\ Mainly used to capture and gather, distinguish and identify the information of objects in the physical world; This layer includes RFID tags, cameras, GPS, sensors, laser scanners, and so on
        \item \textbf{Network Layer}\\ Used to forward packets over a reliable communication medium
        \item \textbf{Application Layer}\\ Processes the data, aggregates various sources and displays it
    \end{enumerate}
    Following that the \textbf{IoT ARCHITECTURE} is explained -
    \begin{enumerate}
        \item \textbf{LoRa}\\ Wireless technology designed to provide the low-power within wide-area networks \textbf{(LPWANs)}
        \item \textbf{SigFox}\\ Ultra-narrowband IoT communications system designed to support IoT deployments over long ranges
    \end{enumerate}
    Then we get to know about the IoT platforms that support Easy integration of new devices and services, Communication between devices, management of different devices and communication protocols, transmission of data flows and the creation of new applications and the scalability of the IoT infrastructure.Under that we discuss the divisions of the platforms and the main features.\\\\
    Finally is the \textbf{weaknesses and security issues} of IoT that can be divided into two categories: 
    \begin{enumerate}
        \item Issues related to the technologies on which IoT is based on
        \item New issues that emerge with IoT deployments
    \end{enumerate}
    And the discussion of the following \textbf{CHALLENGES FACED} - Networking and transport Issues, Security Issues, Heterogeneity Issues, Denial of Services and Big Data Management\\}
    
    \section*{2. KEY CONTRIBUTIONS FROM THE AUTHOR :}
    
    \textit{The authors provide a comprehensive discussion of the topic, discussing the potential benefits and challenges of implementing IoT technologies in smart cities.\\\\They also provide examples of how IoT technologies have been used in smart cities around the world, which helps to understand the real-world application of the technology.\\\\
    The authors also discuss the challenges and future directions of IoT technologies for smart cities, including privacy and security concerns, and the need for standardization and interoperability of different IoT systems.\\\\Additionally, the paper provides insight into how IoT technologies are currently being used in smart cities around the world, and the potential benefits that these technologies can bring to society.\\}

    \section*{3. MY VIEWS ON THE TOPIC :}

    \textit{\begin{itemize}
        \item IoT technologies have the potential to create smart cities that are more efficient, sustainable, and livable for citizens.
        \item IoT technologies can be used in various areas of smart cities, including traffic management, air quality monitoring, energy efficiency, and big data analysis.
        \item Privacy and security are major concerns when using IoT technologies in smart cities and need to be addressed to fully realize their potential.
        \item IoT technologies can bring significant benefits to society in terms of improving the quality of life for citizens, reducing pollution, and making cities more efficient.
        \item The potential of IoT technologies in smart cities is still largely untapped and there is a need for more research and development to fully realize its potential.
    \end{itemize}}
    
    \section*{4. AGREEMENTS :}
    
    \textit{\begin{itemize}
        \item Working of different systems together using standardization and interoperability
        \item The potential of IoT technologies in smart cities is still largely untapped and there is a need for more research and development to fully realize its potential
    \end{itemize}}

    \section*{5. PITFALLS :}

    \textit{\begin{itemize}
        \item Explanation of each layers could have been even more elaborate and intricate.
        \item The paper discusses how different IoT technologies have been used in smart cities , but it does not provide a comprehensive analysis on the effectiveness of these implementations.
    \end{itemize}}

    \section*{6. PAPER DETAILS :}
     \textit{\textbf{PAPER :}Internet of Things (IoT) Technologies for Smart Cities}\\
     \textit{\textbf{AUTHORS :}Badis HAMMI, Rida KHATOUN, Sherali ZEADALLY, Achraf FAYAD, Lyes KHOUKHI}\\
\end{document}

